\documentclass[a3paper,8pt]{extarticle}
\usepackage[utf8]{inputenc}

\usepackage{fancyhdr}

\usepackage[pdftex]{graphicx} % Required for including pictures
\usepackage[pdftex,linkcolor=black,pdfborder={0 0 0}]{hyperref} % Format links for pdf
\usepackage{calc} % To reset the counter in the document after title page
\usepackage{enumitem} % Includes lists

\usepackage{textcomp}
\usepackage{eurosym}

\usepackage{ dsfont } % font za množice
% tabele
\usepackage{array}
\usepackage{wrapfig}

\usepackage{tikz,forest}
\usetikzlibrary{arrows.meta}

\frenchspacing % No double spacing between sentences
\setlength{\parindent}{0pt}
\setlength{\parskip}{1.7em}

\usepackage{mathtools}
\usepackage{blkarray, bigstrut} %


\usepackage{amssymb,amsmath,amsthm,amsfonts}
\usepackage{multicol,multirow}
\usepackage{calc}
\usepackage{ifthen}
\usepackage{tabularx}
\usepackage[landscape]{geometry}
\usepackage{listings}
\usepackage{inconsolata}
%\usepackage[colorlinks=true,citecolor=blue,linkcolor=blue]{hyperref}
%\usepackage{accents}

\newcommand{\vect}[1]{\accentset{\rightharpoonup}{#1}}

\ifthenelse{\lengthtest { \paperwidth = 11in}}
    { \geometry{top=.5in,left=.5in,right=.5in,bottom=.5in} }
	{\ifthenelse{ \lengthtest{ \paperwidth = 297mm}}
		{\geometry{top=1cm,left=1cm,right=1cm,bottom=1cm} }
		{\geometry{top=1cm,left=1cm,right=1cm,bottom=1cm} }
	}
\pagestyle{empty}
\makeatletter
\renewcommand{\section}{\@startsection{section}{1}{0mm}%
                                {-1ex plus -.5ex minus -.2ex}%
                                {0.5ex plus .2ex}%x
                                {\normalfont\Large\bfseries}}
\renewcommand{\subsection}{\@startsection{subsection}{2}{0mm}%
                                {-1explus -.5ex minus -.2ex}%
                                {0.5ex plus .2ex}%
                                {\normalfont\large\bfseries}}
\renewcommand{\subsubsection}{\@startsection{subsubsection}{3}{0mm}%
                                {-1ex plus -.5ex minus -.2ex}%
                                {1ex plus .2ex}%
                                {\normalfont\normalsize\bfseries}}
\makeatother
\setcounter{secnumdepth}{0}
%\setlength{\parindent}{0pt}
%\setlength{\parskip}{0pt plus 0.5ex}

% listings okolje za psevdo kodo
\lstnewenvironment{koda}[1][] %defines the algorithm listing environment
{   
    \lstset{ %this is the stype
        mathescape=true,
        %basicstyle=\scriptsize, 
		columns=flexible,
        keywordstyle=\bfseries\em,
        keywords={,vhod, izhod, zacetek, konec, koncamo, ponavljaj, dokler, ce, vrni, za, vsak, vse, v, sicer,} %add the keywords you want, or load a language as Rubens explains in his comment above.
        xleftmargin=.1\textwidth,
		tabsize=4,
		%frame=leftline,xleftmargin=5pt,xrightmargin=5pt,framesep=5pt,
		%inputencoding = utf8,
		extendedchars = true,
		literate={ž}{{\ˇz}}1 {š}{{\ˇs}}1 {č}{{\ˇc}}1 {Ž}{{\ˇZ}}1 {Š}{{\ˇS}}1 {Č}{{\ˇC}}1,
        #1 % this is to add specific settings to an usage of this environment (for instnce, the caption and referable label)
    }
}
{}
% -----------------------------------------------------------------------


\newcolumntype{C}[1]{>{\centering\arraybackslash$}m{#1}<{$}}
\newlength{\mycolwd}                                         % array column width
\settowidth{\mycolwd}{$e^{-\frac{i}{\hbar}|A|t}$}% "width" of $e^{-\frac{i}{\hbar}|A|t$; largest element in array
\begin{document} 

\begin{multicols}{5}
\setlength{\premulticols}{1pt}
\setlength{\postmulticols}{1pt}
\setlength{\multicolsep}{1pt}
\setlength{\columnsep}{2pt}

\section*{Teorija aproksimacije}
\begin{align*}
    X\quad \dots& \quad \text{vekt. prostor katerega el. aproksimiramo} \\
    S \subseteq X \quad \dots& \quad  \text{podprostor aproksimantov} \\
    A: X \to S \quad \dots& \quad \text{operacijska shema (operator)}
\end{align*}

Prileganje aproksimanta ocenimo z normo:
\begin{itemize}
    \item \textbf{Neskončna norma} 
    \[ \| f \|_{\infty, [a,b]} = \max_{x \in [a, b]} | f(x) | \]
    \textit{Numerični približek:} na intervalu $[a, b]$ izberemo konkčno mnogo točk $a \leq x_0 < x_1 < \dots < x:n \leq b$. 
    \[ \| f \|_{\infty, [a,b]} = \max_{i = 0, \dots, n} | f(x_i) | \]
    \item \textbf{Druga norma} 
    \begin{align*}
        \| f \|_{2} &= \sqrt{\langle f, f \rangle} &
        \langle f, g \rangle &= \int_a^b f(x) g(x) \rho(x) dx 
    \end{align*}
    Standardni skalarni produkt: $\rho \equiv 1$.

    \textit{Numerični približek:} vzamemo diskretni skalarni produkt.
    Na intervalu $[a, b]$ izberemo konkčno mnogo točk $a \leq x_0 < x_1 < \dots < x:n \leq b$.
    \[\langle f, g \rangle = \sum_{i=0}^n f(x_i) g(x_i) \rho(x_i) dx\]
\end{itemize}

\subsection*{Optimalni aproksimacijski problem}
Za $f \in X$ iščemo aproksimant $\hat{f} \in S$, da je
\[ \| f - \hat{f} \| = \min_{s \in S} \| f - s \| \]

\subsection*{Aproksimacija po metodi najmanjših kvadratov}
Naj bo $X$ vektorski prostor nad $\mathbb{R}$ s skalarnim produktom $\langle \cdot, \cdot \rangle$ in normo $\| \cdot \|_2 = \sqrt{\langle \cdot, \cdot \rangle}$ 
\[ S = \text{Lin}\{l_1, l_2, \dots, l_n\} \subseteq X \]

Iščemo \textbf{element najboljše aproksimacije po MNK} $f^* \in S$, da $\| f - f^* \| = \min_{s \in S} \| f - s \| $

\textit{Izrek:} $f^*$ je el. najboljše aproksimacije po MNK $\iff$ $f-f^* \perp S$ $\iff$
$f-f^* \perp l_i \quad \forall i = 1,\dots n$

\[ f^* = \alpha_1 l_1 + \dots + \alpha_n l_n\]

Iz zgornjega izreka sledi:
\begin{align*}
\langle f - f^*, l_i \rangle &= 0 \quad \forall i\\
\langle f - \sum_{j=1}^n \alpha_j l_j, l_i \rangle &= 0 \quad \forall i\\
\langle f, l_i\rangle - \sum_{j=1}^n \alpha_j \langle  l_j, l_i \rangle &= 0 \quad \forall i\\
\end{align*}
V matrični obliki:
\[
    \underbrace{\begin{bmatrix}
        \langle l_1, l_1 \rangle & \dots & \langle l_n, l_1 \rangle \\
        \vdots & \ddots & \vdots \\
        \langle l_1, l_n \rangle & \dots & \langle l_n, l_n \rangle
    \end{bmatrix}}_{\text{Grammova matrika $G$}}
    \begin{bmatrix}
        \alpha_1 \\
        \vdots \\
        \alpha_n
    \end{bmatrix}
    =
    \begin{bmatrix}
        \langle f, l_1 \rangle \\
        \vdots \\
        \langle f, l_n \rangle
    \end{bmatrix}
\]

$G$ je simetrična pozitivno definitna matrika. Numerično tak sistem rešimo z razcepom Choleskega.


Reševanje sistema linearnih enačb se izognemo tako, da bazo za $S$ ortonormiramo. Tedaj je $G = I$ in
\[ f^* = \sum_{i=1}^n \langle f, l_i \rangle l_i \]

\subsubsection*{Gram-Schmidtova ortogonalizacija}
Definirajmo projekcijo vektorja $v$ na $u$
\[\textmd{proj}_u(v) = \frac{\langle v,u \rangle}{\langle u,u \rangle}u\]
Če želimo \emph{orotogonalizirati} $k$ linearno neodvisnih vektorjev $v_1, ..., v_k$, uporabimo postopek:
\begin{equation*}
    \begin{aligned}
    u_1 &= v_1 \\
    u_2 &= v_2 - \textmd{proj}_{u_1}(v_2) \\
    u_3 &= v_3 - \textmd{proj}_{u_1}(v_3) - \textmd{proj}_{u_2}(v_3) \\
    &\; \ \vdots \\
    u_k &= v_k - \sum_{j=1}^{k-1} \textmd{proj}_{u_j}(v_k)
    \end{aligned}
\end{equation*}


\subsection*{Enakomerna aproksimacija zveznih funkcij s polinomi}
Za dano funkcijo $f \in \mathcal{C}([a,b])$ iščemo \textbf{polinom najboljše enakomerne aproksimacije} $p^* \in \mathbb{P}_n$, za katerega velja
\[ \| f - p^* \|_{\infty, [a,b]} = \min_{p \in \mathbb{P}_n} \| f - p\|_{\infty, [a,b]} = \min_{p \in \mathbb{P}_n} \max_{x \in [a, b]} | f(x) - p(x) | \]

\textit{Izrek:} Naj bo $f \in \mathcal{C}([a, b])$. Če je polinom $p \in \mathbb{P}_n$ takšen,
da \textbf{residual} $r = f - p$ doseže svojo normo $\| r \|_{\infty, [a,b]}$ alternirajoče:
$n+2$ tokčah $x_i$, $a \leq x_0 < \dots < x_{n+1} \leq b$ 
\[ r(x_i)r(x_{i+1}) < 0 \qquad \forall i=0, \dots, n \]
potem je $p$ polinom najboljše enak. aproks. za $f$ na $[a, b]$.

\subsubsection*{Remesov postopek}
Vhodni podatki: funkcija $f$, interval $[a,b]$, stopnja $n$, toleranca $\varepsilon$

Izberi množico točk $E_0 = \{ x_i,\ a \leq x_0 < \dots < x_{n+1} \leq b \}$.

\textbf{Minimaks} za funkcijo $f$ na $E$ je
\[M_n(f, E) = \min_{p \in \mathbb{P}} \max_{x \in E} | f(x) - p(x) | = |m|\]

Ponavljaj $k = 0, 1, 2, \dots$:
\begin{itemize}
    \item Poišči polinom $p_k^* \in \mathbb{P}_n$, ki zadošča pogoju:
    \[ f(x_i) - p_k^* = (-1)^i m \qquad \forall i = 0, 1, \dots, n+2 \]
    \item Poišči ekstrem residuala $r_k = f - p_k^*$. To je $u \in [a,b]$, da
    \[ |r_k(u)| = \| r_k \|_{\infty, [a,b]} \]
    \item Če je $|r_k(u)| - |m| < \epsilon$, potem končaj in vrni $p^* = p_k^*$. 
    Sicer pa naredimo zamenjavo točk v množici $E_k$ z $u$ tako, da ogranimo alternacijo residuala.
\end{itemize}

\subsection*{Interpolacija}
Podane so vrednosti izbarne funkcije $f$ v $n+1$ paroma različnih točkah $x_i$ (\textbf{interpolacijske točke}),
iščemo neko preprostejšo funkcijo $g$ (\textbf{interpolacijska funkcija}), ki zadošča pogoju:
\[ f(x_i) = g(x_i) \qquad i = 0, 1, \dots, n \]

\subsection*{Polinomska interpolacija}
$f \in \mathcal{C}([a,b])$, $a \leq x_0 < \dots < x_n \leq b$. Iščemo polinom
\[ p(x) = a_0 + a_1 x + a_2 x^2 + \dots + a_n x^n \]
ki zadošča pogoju $f(x_i) = p(x_i)$ za $i = 0, 1, \dots, n$. 
\[
\underbrace{\begin{bmatrix}
    1 & x_0 & x_0^2 & \dots & x_0^n \\
    1 & x_1 & x_1^2 & \dots & x_1^n \\
    \vdots & &        &       & \vdots \\
    1 & x_n & x_n^2 & \dots & x_n^n \\
\end{bmatrix}}_{\text{Vandermondova matrika $V$}}
\begin{bmatrix}
    a_0 \\
    a_1 \\
    \vdots \\
    a_n
\end{bmatrix}
=
\begin{bmatrix}
    f(x_0) \\
    f(x_1) \\
    \vdots \\
    f(x_n) \\
\end{bmatrix}
\]
\[ det V = \prod_{0 \leq j < i \leq n} (x_i - x_j) \neq 0\]
Vidimo, da je polinom $p$ enolično določen.

\subsubsection*{Lagrangeova oblika interpolacijskega polinoma}
\[ l_{i, n}(x) = \prod_{\substack{j=0 \\ j\neq i}}^n \frac{x-x_j}{x_i - x_j} \qquad i = 0, \dots, n \]

% ni ključno
Velja:
\[ 
l_{i,n}(x_j) = 
\begin{cases}
    1 & i = j \\
    0 & \text{sicer}
\end{cases}
\]
\textit{Lema:} $l_{i, n}$ so baza za $\mathbb{P}_n$.
%

Interpolacijski polinom v lagrangeovi obliki:
\[ p(x) = \sum_{i=0}^n f(x_i) l_{i,n}(x)\]

\subsubsection*{Newtonova oblika zapisa interpolacijskega polinoma}

Za bazo izberemo prestavljene potence: $1$, $(x-x_0)$, $(x-x_0)(x-x_1)$, \dots  


\textbf{Deljena diferenca} $[x_0, x_1, \dots, x_k] f$ je vodilni koeficient interpolacijskega polinoma
stopnje $k$, ki se s funkcijo $f$ ujema v točkah $x_0, x_1, \dots, x_k$. Sledi:
\[ p_k(x) = p_{k-1}(x) + [x_0, x_1, \dots, x_k] f (x-x_0)\dots (x-x_{k-1})\]
Rekurzivna zveza:
\begin{multline*}
    [x_i, x_{i+1}, \dots, x_{i+k}]f = \\
    = \begin{cases}
        \frac{f^{(k)}(x_i)}{k!} & {\scriptstyle x_i = x_{i+1} = \dots = x_{i+k}} \\
        \frac{[x_{i+1}, \dots, x_{i+k}]f- [x_i, \dots, x_{i+k-1}]f}{x_k-x_0} & {\scriptstyle x_i \neq x_{i+k} }
    \end{cases} 
\end{multline*} 

Newtonova oblika zapisa je torej:
\[ p(x) = \sum_{i=0}^n [x_0, \dots, x_i]f (x-x_0)\dots (x-x_{i-1}) \]

\textit{Če želimo, da se polinom n neki toči ujema tudi v $k$-tem odvodu, to točko $k$-krat ponovimo.}

\subsubsection*{Računanje vrednosti interpolacijskega polinoma v Newtonovi obliki}
\[ p(x) = d_0 1 + d_1(x-x_0) + d_2(x-x_0)(x-x_1) + \dots \]

\begin{koda}
vhodni podatki: $x_0, \dots, x_n, \ d_0, \dots, d_n, \ x$
$v_n \leftarrow d_n$
za $i = n-1, \dots, 0$:
    $v_i \leftarrow d_i + (x-x_i) v_{i+1}$

vrni $v_0$
\end{koda}

\subsubsection*{Kako izbrati točke na $[a,b]$?}
\begin{itemize}
    \item \textbf{Ekvidistantne točke}: $h = \frac{b-a}{n}$, $x_i = a + ih$
    \item \textbf{Izbira pri kateri je dosežen minimum}
    \item \textbf{Čibiševe točke}
    \[ x_i = \frac{a+b}{2} + \frac{a-b}{2}\cos\left(\frac{2i+1}{2n+2} \pi\right) \qquad i = 0, \dots, n\]
\end{itemize}

\end{multicols}
\end{document}